\documentclass{llncs}
\usepackage[spanish]{babel}
\usepackage[utf8]{inputenc}
\usepackage{llncsdoc}
\usepackage{graphicx}
\usepackage{listings}

\title{Análisis de CBMC y su funcionamiento.}
\author{Mauricio D. Mazuecos Perez, Facundo Molina Heredia y Hernán J. Maina}
\institute{Facultad de Matemática, Astronomía, Física y Computación\\
Universidad Nacional de Córdoba}

%Empieza el documento!!!
\begin{document}
\maketitle

%=============Resumen=============
\begin{abstract}
    En este documento presentamos un análisis de la herramienta CBMC, que usa
    model checking acotado (BMC, por sus siglas en inglés) para
    programas de C y C++. La idea del mismo es aportar datos
    relevantes de la herramienta y su desempeño en distintas aplicaciones. Luego
    de leer este documento, el lector tendría la información necesaria para
    saber si CBMC es el model checker que necesita.

\end{abstract}


%=============Introducción=============
\section{Introducción}
    La herramienta CBMC (C Bounded Model Checker) fue nombrada por primera vez
    en el 2003 en un paper acerca de consistencia en comportamiento de programas
    en C y Verilog\cite{firstcbmc} y presentada formalmente en el
    2004\cite{cbmcpaper}. Originalmente se creó para corroborar la consistencia
    del modelo de nuevos dispositivos. Estos dispositivos estaban usualmente
    descritos en un lenguaje de programacíon como ANSI-C, y su implementación
    escrita en Verilog.\\
    \indent En este documento presentaremos un análisis de la herramienta,
    explicaremos su funcionamiento y mostraremos unos pequeños casos de uso de
    la misma. Compararemos la herramienta a otras existentes y daremos una breve
    reflexión de la utilidad y los campos de acción de la misma.\\
    \indent El documento se organizará de la siguiente manera: en la siguiente
    sección presentaremos algunos conceptos necesarios para entender el
    funcionamiento de las técnicas empleadas por CBMC; luego explicaremos la
    mecánica y el funcionamiento de CBMC; en la siguiente sección expondremos
    comparaciones con otros comprobadores de modelos y el uso de la
    herramienta, y finalizaremos con una breve conclusión.

%=============Preliminares=============
\section{Preliminares}
    Para poder introducir al lector a la mecánica de CBMC, primero
    introduciremos una serie de conceptos necesarios para la comprensión de la
    herramienta.\\ \indent
    CBMC es una implementación de BCM. BMC \cite{bmcpaper}(Bounded Model
    Checking) es una técnica de comprobación de modelos que trata al problema
    bajo consideración como una máquina de estados finitos, $M$, con un umbral
    de $k$ estados para la ejecución. Tal máquina de estados está compuesta de n
    estados $s_{1}$...$s_{n}$ y m transiciones, donde cada transición
    $\varphi_{i}$ es una terna (s,$\alpha$,s') (i.e. si s es el estado actual y
    la condición $\alpha$ se cumple, entonces transiciona al estado s').  La
    idea básica detrás de BCM es, dada una determinada propiedad $P$, buscar un
    contra ejemplo para $P$ con $k$ transiciones de estado como máximo. Si no lo
    encuentra, aumenta el umbral en uno hasta que bien encuentre un contra
    ejemplo o el problema se vuelva intratable.\\
    \indent Para representar el espacio de estados y posteriormente formar la
    fórmula que será pasada a un SAT Solver (solucionador de problemas de
    satisfacibilidad proposicional), CBMC transforma el programa a un
    CFG (Grafo de Control de Flujo, por sus siglas en inglés). Un CFG es una
    representación, usando notación de grafos, de todas las posibles trazas
    de un programa durante su ejecución.\\
    \indent Finalmente, CBMC genera una fórmula en CNF para pasarla a un
    SAT Solver y comprobar una propiedad.  En lógica booleana, una fórmula está
    en CNF (Forma Normal Conjuntiva, por sus siglas en inglés) si es una
    conjunción de cláusulas. Los únicos conectivos que una fórmula puede tener
    en CNF son $\wedge$, $\vee$ y $\neg$.\\
    \indent En la siguiente sección se tratará la implementación y los detalles
    técnicos de la herramienta.\\

%===========CBMC Generalidades ================
\section{CBMC - Características y métodos}
    BMC transforma los programas de C y C++ en un CFG, con el cual representará
    el espacio de estados, incluyendo solo aquellos estados alcanzables en una
    ejecución. 
    La idea detrás de esto es tomar caminos en el CFG hasta una aserción y
    construir la fórmula correspondiente a tal camino. Esta fórmula luego es
    pasada a un Sat Solver, que nos dirá si la misma es satisfacible o no, i.e.
    si esa ejecución es alcanzable o no.\\
    % Cómo trabaja u
    \indent Dado un sistema de transición $A=(S,s_{0},T)$ (donde $S$ es un
    conjunto de estados, $s_{0}$ un estado inicial y $T$ una relación de
    transición), para evitar la explosión  exponencial de caminos, CBMC
    simplemente concatena la relación $T$ a cada paso. Es decir, busca una
    sucesión de k+1 estados $s_{1}...s_{k}$ tal que: $s_{0}$ sea el estado
    inicial; para cada dos estados adyascentes $s_{i}...s_{i+1}$ exista una
    transición $T$, y $P$ no se satisfaga en $s_{k}$. Luego, las asignaciones
    que satisfacen la ecuación (\ref{eq1}) son trazas en $A$. Si queremos
    corroborar una propiedad $\textbf{AG}p$, solo se tiene que encontrar un
    $s_{i}$ que satisfaga $\neg p$. Encontrar una asignación que satisfaga la
    ecuación (\ref{eq2}), es encontrar un contraejemplo para la propiedad
    $\textbf{AG}p$.\\

    % Ecuaciones usadas por CBMC
    \begin{equation}
        S_{0}(s_{0}) \wedge \bigwedge_{i=0}^{k-1} T(s_{i},s_{i+1})
        \label{eq1}
    \end{equation}
    \begin{equation}
        S_{0}(s_{0}) \wedge \bigwedge_{i=0}^{k-1} T(s_{i},s_{i+1}) \wedge
        \bigvee_{i=0}^{k} \neg p(s_{i})
        \label{eq2}
    \end{equation}

    \indent Para optimizar este proceso, las partes no alcanzables del
    programa no son generadas durante la fase de desenrrollamiento.

    \subsection{Implementación}
        Obviamente, la implementación en sí no es tan simple, y requiere abordar
        muchos problemas pequeños, como bucles, azúcares sintácticos, modelo
        aritmético, etc. 
        \\ \indent En una primera etapa el programa es simplificado removiendo
        todos los azucares sintácticos y estructuras en el código (i.e. se
        sustituyen $i++$ por $i=i+1$, ciclos $for$ por $while$, etc).
        Luego de ello, se procede a 'desenrrollar' todos los bucles del código,
        i.e. se sustituyen $while$(condición) por $if$(condición), hasta una
        profundidad $k$, en la cual se introduce un $assert$($\neg$condición).
        \\ \indent Pasada esta etapa, lo siguiente es resolver las multi
        asignaciones de variables ya que pueden ocasionar problemas
        semánticos relacionados. Para ello cuando una variable es asignada más
        de una vez, se agrega una variable nueva para cada nueva asignación.
        Al código resultante, tras completar las anteriores etapas, se lo
        denomina SSA-Program (Asignaciones Estáticas Simples, por sus siglas en
        ingles) y es el código intermedio y precursor que será utilizado
        en la construcción de las fórmulas CNF para la corroboración de
        propiedades, de la manera en que anteriormente se ha descripto.

        \subsection{Fases de verificación}
          A contianuación explicaremos el flujo de la verificación.
            % DESCOMENTEN LA IMAGEN DESPUÉS
            %\begin{figure}[h]
                %\centering
                %\includegraphics[width=0.7\textwidth]{CBMC.png}
                %\caption{Esquema de funcionamiento de CBMC.}
                %\label{fig:cbmc}
            %\end{figure}
            \begin{enumerate}
                \item El programa en C se introduce en el motor de análisis
                    de CBMC junto con la propiedad a satisfacer $P$ y un cota
                    $k$. \\
                \item Se genera una fórmula CNF mediante la adición de
                    variables intermedias, junto con un término que describe la
                    negación de la propidad ($C \wedge \neg P$).\\
                \item La fórmula en  CNF resultante se introduce en un SAT
                    Solver. Si la ecuación se satisface, encontramos una
                    violación de la propiedad, caso contario, si no se
                    satisface, la propiedad se satisface.\\
                \item En caso de haber encontrado un contraejemplo, se
                    vuelve a traducir a "lenguaje de programa" de manera de
                    %ARREGLAR, SE PASA A ESE LENGUAJE ILEGIBLE QUE TE TIRA EN EL
                    %OUTPUT
                    facilitar la visualización del error al programador.\\
                \item En el caso de que el SAT solver crahsee, dependiendo
                    de que haya generado el crasheo, el programa respondera de
                    diversas maneras. Lo mas probable es que simplemente, se
                    clave. %CONTROLAR QUE NO SEA FRUTA
            \end{enumerate}

         La herramienta no retorna \emph{falsos negativos}, i.e. si encuentra un
         contraejemplo que no satisface la propiedad en estudio, éste es
         certero; pero sí puede otorgar \emph{falsos positivos}, ya que CBMC
         comprueba la satisfacción de la ecuación (\ref{eq2}) para ejecuciones
         de trazas de a lo sumo $k$ estados, lo cual nada asegura la
         insatisfactibilidad de la ecuación mas allá del $k$-ésimo estado.


% ============== Descripción del lado del usuario ===========
    \subsection{Descripción de la herramienta del lado del usuario}
        CBMC puede ser encontrada tanto para ser utilizada mediante línea de
        comandos como por interfaz gráfica. Una posible opción para ésta última
        modalidad, es por medio de CProver suite, un plug-in para Visual Estudio
        que le brinda soporte \cite{webcprover}.\\
        \indent La herramienta trabaja directamente sobre el código del
        programa, específicamente sobre lenguaje ANSI-C y derivados. El lenguaje
        de especificación de propiedades es asercional.\\ \indent Cabe destacar
        que en caso de encontrar un error en el programa, se reporta un
        contraejemplo.
        Para la opción de linea de comandos, este se visualiza mediante
        una sucesión de variables y los correspondientes valores que hacen falsa
        la propiedad en estudio. Además la misma no permite simulación.

    %============= Funcionalidades =============
    \subsection{Funcionalidades}
        Con CBMC se pueden verificar tanto propiedas particulares de interés
        semántico, como accesos correctos a arreglos y ausencia de memoria sin
        liberar entre otras propiedades deseables que todo programa debe
        satisfacer. CMBC también comprueba propiedades de una lista
        autogenerada en base a un previo análisis estático. Estas propiedades
        autogeneradas no corresponden necesariamente a bugs, sino que,
        comprueban posibles fallos. CBMC cuenta con comandos para ver las
        propiedades autogeneradas que verifica. Además se puede pedir que luego
        de verificar ciertas propiedades, CBMC devuelva una traza de
        contraejemplo para las propiedades que fallaron en la comprobación.
        A veces no queremos comprobar todo un programa, sino solamente una
        función, o un módulo, en estos casos también podemos usar CBMC, tanto
        con propiedades autogeneradas como con propiedades específicas.

    \subsection{Nota sobre compiladores y librerías de ANSI-C}
        La mayoría de los programas en C usan funciones provistas por
        librerias.
        Las comprobaciones para dichos programas requieren de dos cosas:
        1) Que los archivos .h de dichas funciones sean provistos;
        2) Y que las definiciones apropiadas de las funciones sean provistas.

%=============Comparaciones con otras herramientas =================
\section{Comparaciones y uso}
    Vamos a comparar CBMC con LLBMC \cite{firstcomp} (Low-Level Bounded Model
    Checker). LLBMC lleva los programas
    de C a una representación intermedia de LLVM (Low-Level Virual Machine)
    que luego es convertido a una
    fórmula lógica, que después de ser simplificada, se pasa por un SMT Solver.
    En un estudio realizado en el Institute for Theoretical Computer Science
    KIT, en Alemania, compararon el desempeño de LLBC con las versiones 3.8 y
    3.9 de CMBC usando benchmarks de distintos tipos. Para compersar las
    diferencias en los ajustes entre herramientas CBMC se corrió con
    \emph{- -bounds-check}, \emph{- -div-by-zero-check}, \emph{-
    -pointer-check}, y \emph{- -overflow-check},
    mientras que LLBMC se corrió con \emph{llvm-gcc} para convertir los programas
    de C a LLVM, sin las optimizaciones del compilador y con la configuración de
    Boolector. Entre los benchmarks había programas en C, C++ y programas con
    bucles no infinitos.
    Como resultado de los 175 benchmarks:
    \begin{itemize}
    \item CBMC 3.9 resolvió satisfactoriamente 119(68\%), tomo mas tiempo del
        permitido en las pruebas o uso mas memoria de la permitida en 8 casos
        (4,6\%), falló en procesar el input en 12 casos (6,9\%) y dio resultados
        incorrectos en 36 casos (20,6\%).
    \item CBMC 3.8 resolvió satisfactoriamente 145(82,9\%), tomó mas tiempo del
        permitido en las pruebas o usó mas memoria de la permitida en 13 casos
        (7,4\%), fallo en procesar el input en 6 casos (3,4\%) y dio resultados
        incorrectos en 11 casos (6,3\%).
    \item LLBMC resolvió satisfactoriamente 172(98,3\%), tomo mas tiempo del
        permitido en las pruebas o uso mas memoria de la permitida en 1 caso
        (0,6\%), fallo en procesar el input en 0 casos y dio resultados
        incorrectos en 2 casos (1,1\%).
    \end{itemize}
    Si comparamos los resultados obtenidos con los tiempos en que cada
    herramienta los obtuvo, podemos apreciar que no solamente LLBMC obtuvo
    menos timeouts/lleno memoria, o dio menos errores, sino que además, obtuvo
    los resultados en menos tiempo que CBMC tanto en la version 3.8 como la
    3.9. En aproximadamente 100 segundos, CBMC 3.9 obtuvo aproximadamente 100
    resultados, CBMC 3.8 obtuvo aproximadamente 140 resultados, mientras que
    LLBMC obtuvo cerca de 180 resultados.

    % ============== Casos de estudio ===========
    \subsection{Uso de la herramienta}
    Para mostrar el uso de la herramienta, tomamos dos programas: uno que
    calcula la prefix sum en dos dimensiones y otro que usa pthreads para
    calcular el producto punto de vectores. Nos centraremos en el uso de la
    herramienta a través de su interfaz de linea de comandos. El código fuente
    estará disponible
    en un repositorio nombrado en las referencias\cite{repository}.\\
    \indent Comenzaremos con el programa \emph{scan2d.c}. CBMC facilita la forma
    de ver las propiedades de un programa. Escribiendo en consola \emph{cbmc
    - -show-properties scan2d.c}  nos dirá las aserciones que el
    programa tiene. Una porción de la salida de esta llamada:\\

    {\fontsize{7}{8}\selectfont
    \begin{lstlisting}
    Generic Property Instrumentation
    Property main.assertion.1:
        file scan2d.c line 104 function main
        free argument is dynamic object
        DYNAMIC_OBJECT(ptr)
    \end{lstlisting}
    }

    \indent Si queremos comprobar que el programa cumple con las propiedades
    descriptas, escribimos en la consola \emph{cbmc scan2d.c}. Al correr esto,
    el programa empezará a analizar todos los posibles estados del programa para
    verificar que las aserciones se cumplen. Esto puede llegar a ser intratable
    si el espacio de estados es muy grande, pero podemos pedir que los bucles se
    corroboren para una cierta cantidad de iteraciones. Corriendo \emph{cbmc -
    -unwind 6 - -no-unwinding-assertions scan2d.c} podemos analizar el programa
    para 6 iteraciones por bucle.\\

    {\fontsize{7}{8}\selectfont
    \begin{lstlisting}
    size of program expression: 4164 steps
    simple slicing removed 22 assignments
    Generated 7 VCC(s), 1 remaining after simplification
    Passing problem to propositional reduction
    converting SSA
    Running propositional reduction
    Post-processing
    Solving with MiniSAT 2.2.0 with simplifier
    23239 variables, 156 clauses
    SAT checker inconsistent: negated claim is UNSATISFIABLE, i.e., holds
    Runtime decision procedure: 0.009s
    VERIFICATION SUCCESSFUL
    \end{lstlisting}
    }

    \indent El flag de \emph{- -no-unwinding-assertions} permite ignorar
    problemas de la  no satisfactibilidad de una propiedad debido a que no tuvo
    suficientes ciclos para encontrar una fórmula que la dé como satisfacible.
    En contraposición, \emph{- -unwinding-assertions} buscará una cota en
    tiempo de ejecución para asegurar de que suficiente desenrollamiento sea
    realizado para probar la propiedad.\\
    \indent Podemos acceder a la fórmula que CBMC ingresará al
    SAT Solver con el flag \emph{- -show-vcc}. Una porción de la misma tiene
    esta forma:\\

    {\fontsize{7}{8}\selectfont
    \begin{lstlisting}
    {-2438} __CPROVER_memory_leak#4 == NULL
    {-2439} __CPROVER_memory_leak#5 == (\guard#3 ? NULL :
    __CPROVER_memory_leak#3)
    {-2440} ptr!0@1#5 == (void *)dynamic_object2
    {-2441} \guard#4 == (__CPROVER_deallocated#5 == (void *)
    dynamic_object2)
    |--------------------------
    {1} !\guard#4
    \end{lstlisting}
    }

    \indent En caso de que un assert no se cumpla en una fórmula, CBMC nos dará
    un contraejemplo, podemos además probar solo una propiedad con el flag
    \emph{- - property} y \emph{- -function} nos dejará probar para una función
    en particular. Usando el programa \emph{dotprod\_mutex.c}, corremos
    \emph{cbmc - -propierty dotprod.assertion.1
    - -function dotprod dotprod\_mutex.c} y nos encuentra un contraejemplo de
    que la propiedad \emph{dotprod.assertion.1} no se cumple en la función
    \emph{dotprod}.

    {\fontsize{7}{8}\selectfont
    \begin{lstlisting}
    Counterexample:

    State 21 file dotprod_mutex.c line 53 thread 0
    ----------------------------------------------------
      arg=NULL
      (0000000000000000000000000000000000000000000000000000000000000000)

    Violated property:
      file dotprod_mutex.c line 58 function dotprod
      assertion arg != NULL
      arg != (void *)0

    VERIFICATION FAILED
    \end{lstlisting}
    }

    \indent Además, CBMC cuenta con flags para verificar acceso a memoria,
    memory leaks, punteros colgantes y puede verificar programas que usan
    pthreads y probar programas concurrentes, modelando cada hilo.

    \subsection{Otros usos}
        Aparte de la aplicación básica de la verificación de programas C, CBMC
        se utiliza para una amplia variedad de aplicaciones, como:
        \begin{itemize}
            \item Explicación de errores: las versiones extendidas de CBMC
                pueden encontrar y explicar la causa de un error.
            \item BMC de programas concurrentes: verificación de programas C
                ejecutados por múltiples subprocesos.
            \item Comprobación de equivalencia: verificando que dos programas
                son equivalentes, en el sentido de que siempre calculan la misma
                salida.
            \item Verificación de programas embebidos y modelos que no son de
                programa: los modelos se formalizan utilizando código C y se
                verifican utilizando CBMC.
            \item Verificación de programas existentes, como los controladores
                de dispositivos Linux y Windows.
            \item Tiempo de ejecución del peor caso: analizar el tiempo de
                ejecución de los programas.
            \item Seguridad: medición de fugas de información en programas,
                búsqueda de errores de seguridad en binarios de Windows, entre
                otros.
        \end{itemize}

% ============== Conclusiones ===========
\section{Conclusiones}
    CBMC es un comprobador de modelos para programas de ANSI-C que usa BMC como
    método para probar propiedades de un programa dado. El mismo genera un CFG
    a partir del código, que luego recorre para generar una fórmula en CNF para
    combinarla con la negación de la propiedad a probar y pasarla a un SAT
    Solver. CBMC puede comprobar una gran cantidad de propiedades además de
    aserciones en un programa, lo cual lo hace útil a la hora de verificar la
    correctitud del mismo. Sin embargo, el espacio de estados puede ser muy
    extenso en algunas aplicaciones, lo cual lo hace muy costoso de emplear sin
    el correcto cuidado de las restricciones del problema.\\
    \indent La herramienta es poderosa, siendo capaz de encontrar numerósos
    problemas en sistemas dada la correcta instrumentación, pero la lectura de
    contraejemplos a propiedades que no se satisfacen puede ser muy compleja a
    simple vista, requiriendo un aprendizaje y familiarización con la misma.
    Cabe destacar que es posible que la herramienta no detecte errores en el
    código, por lo que esta no podrá afirmar la correctitud del
    programa completo, sino, hasta una cierta profundidad.\\
    \indent Si bien existen herramientas más eficientes, CBMC
    es una muy buena elección para adentrarse en el BMC y aprender acerca de
    ello, ya que facilita las estructuras internas con las que trabaja.

%=============Referencias=============
\begin{thebibliography}{9}
    \bibitem{firstcbmc}
        Edmund Clark, Daniel Kroening y Karen Yorav. \textit{Behavioral
        Consistency of C and Verilog Programs Using Bounded Model Checking.}
    \bibitem{cbmcpaper}
        Edmund Clarke, Daniel Kroening y Flavio Lerda. \textit{A Tool for
        Checking ANSI-C Programs.}
    \bibitem{bmcpaper}
        Armin Biere, Alessandro Cimatti, Edmund M. Clarke, Ofer Strichman y
        Yunshan Zhu. \textit{Bounded Model Checking.}
    \bibitem{cbmccomp}
        Daniel Kroening y Michael Tautschnig. \textit{CBMC – C Bounded Model
        Checker (Competition Contribution).}
    \bibitem{firstcomp}
        Florian Merz, Stephan Falke, and Carsten Sinz. \textit{LLBMC: Bounded
        Model Checking of C and C++ Programs Using a Compiler IR.} Institute
        for Theoretical Computer Science Karlsruhe Institute of Technology
        (KIT) Germany.
    \bibitem{webcprover}
        Sitio web de plug-in CProver suite:
        http://www.cprover.org/visual-studio
    \bibitem{repository}
        Código fuente de los programas usados en este documento:
        https://github.com/maurygreen/CBMC-c-digo-de-ejemplos.
\end{thebibliography}

\end{document}
